\documentclass[dvipdfmx]{jsarticle}
\usepackage{minted}
\usepackage{url}
\author{4EC 中野 将生}
\date{2018.5.30}
\title{データの探索}
\begin{document}
	\maketitle
	\section{概要}
		以下のアルゴリズムについて比較した。
		\begin{itemize}
			\item 線形探索
			\item 線形探索(番兵法)
			\item 二分探索
			\item ハッシュ法(チェイン法)
			\item ハッシュ法(オープンアドレス法)
			\item 幅優先探索(DFS)
			\item 深さ優先探索(WFS)
		\end{itemize}
	\section{実装}
		線形探索、線形探索(番兵法)、二分探索はそれぞれ配列で、DFSとWFSは木で実装した。
		ソースコードの全体は\url{https://github.com/namachan10777/algorithm-7}にある。
		\subsection{ソースコード}
			\subsubsection{線形探索、線形探索(番兵法)、二部探索}
				\inputminted{d}{./src/algorithms/linear.d}
			\subsubsection{幅優先探索、深さ優先探索}
				\inputminted{d}{./src/algorithms/tree.d}
			\subsubsection{ハッシュ法}
				\inputminted{d}{./src/algorithms/hash.d}
	\section{実行結果}
	\section{環境}
		\subsection{実行環境}
			\subsubsection{CPU}
				% TODO やれ
			\subsubsection{OS}
				Arch Linux \\
				カーネルバージョン 4.16.11-1-ARCH
			\subsubsection{RAM}
				% TODO 詳しく
				8 GB
		\subsection{ビルドツール}
			\subsubsection{OS}
				Arch Linux \\
				カーネルバージョン 4.16.11-1-ARCH
			\subsubsection{コンパイラ}
				DMD64 D Compiler v2.080.0-dirty
			\subsubsection{ビルドツール}
				DUB version 1.9.0, built on May  3 2018
			\subsubsection{ビルドオプション}
				--release
\end{document}
